\documentclass[aspectratio=169]{beamer}
\usetheme{Singapore}
\usecolortheme{default}

% Packages
\usepackage{amsmath, amsfonts, amssymb}
\usepackage{graphicx}
\usepackage{booktabs}
\usepackage{array}
\usepackage{multicol}

% Color definitions
\definecolor{kentech_blue}{RGB}{0,51,102}
\definecolor{kentech_orange}{RGB}{255,102,0}
\definecolor{safety_green}{RGB}{34,139,34}
\definecolor{warning_red}{RGB}{220,20,60}

% Custom theme colors
\setbeamercolor{title}{fg=white,bg=kentech_blue}
\setbeamercolor{frametitle}{fg=white,bg=kentech_blue}
\setbeamercolor{structure}{fg=kentech_blue}
\setbeamercolor{block title}{bg=kentech_blue,fg=white}
\setbeamercolor{block body}{bg=kentech_blue!10}

% Title slide information
\title[BAC Prediction Model]{Blood Alcohol Concentration Prediction Model}
\subtitle{Implementation using Fractional Differential Equations and Mittag-Leffler Functions}
\author[G2 Team]{Hyunjun Jang \and Minyeop Jin \and Sangsu Lee \and Seojin Choi}
\institute[KENTECH]{Korea Institute of Energy Technology (KENTECH)\\
Engineering Mathematics 1 - Group 2}
\date{\today}

% Custom commands
\newcommand{\highlight}[1]{\textcolor{kentech_orange}{\textbf{#1}}}
\newcommand{\safety}[1]{\textcolor{safety_green}{\textbf{#1}}}
\newcommand{\warning}[1]{\textcolor{warning_red}{\textbf{#1}}}

\begin{document}

% Title slide
\begin{frame}
    \titlepage
\end{frame}

% Table of Contents
\begin{frame}{Presentation Overview}
    \tableofcontents
\end{frame}

\section{Introduction \& Problem Statement}

\begin{frame}{Project Motivation}
    \begin{columns}
        \begin{column}{0.7\textwidth}
            \begin{block}{Real-World Problem}
                \begin{itemize}
                    \item \warning{42 drunk driving incidents daily} in South Korea (2019-2023)
                    \item \warning{75,950 alcohol-related accidents} over 5 years
                    \item \warning{1,161 fatalities, 122,566 injuries}
                    \item Peak incidents: Thursday/Friday 10PM-midnight
                \end{itemize}
            \end{block}
            
            \begin{block}{Young Adults' Challenge}
                \begin{itemize}
                    \item Lack of understanding of personal alcohol tolerance
                    \item Trial-and-error approach leads to accidents
                    \item Need for \highlight{scientific prediction method}
                \end{itemize}
            \end{block}
        \end{column}
        
        \begin{column}{0.3\textwidth}
            \begin{center}
                \textcolor{kentech_blue}{\Large \textbf{BAC}}\\
                \textcolor{kentech_blue}{\Large \textbf{Calculator}}\\
                \vspace{0.5cm}
                \textcolor{kentech_orange}{\textbf{Fractional}}\\
                \textcolor{kentech_orange}{\textbf{Model}}
            \end{center}
        \end{column}
    \end{columns}
\end{frame}

\begin{frame}{Problem Statement \& Objectives}
    \begin{block}{Traditional Model Limitations}
        \begin{itemize}
            \item Classical first-order kinetic models assume \highlight{constant rates} $k_1, k_2$
            \item Fails to capture \highlight{physiological memory effects}
            \item Ignores \highlight{non-local dynamics} in alcohol metabolism
            \item Inadequate for personalized tolerance prediction
        \end{itemize}
    \end{block}
    
    \vspace{0.5cm}
    
    \begin{block}{Our Solution: Fractional Differential Equations}
        \begin{enumerate}
            \item Implement \highlight{Caputo fractional derivatives} to model memory effects
            \item Use \highlight{Mittag-Leffler functions} for realistic BAC dynamics
            \item Develop \highlight{multi-platform applications} (GUI, Web, CLI)
            \item Predict recovery times for safe driving thresholds
        \end{enumerate}
    \end{block}
\end{frame}

\section{Mathematical Foundation}

\begin{frame}{Classical vs. Fractional Models}
    \begin{columns}
        \begin{column}{0.5\textwidth}
            \begin{block}{Classical Two-Compartment Model}
                \begin{align}
                    \frac{dA}{dt} &= -k_1 A(t) \\
                    \frac{dB}{dt} &= k_1 A(t) - k_2 B(t)
                \end{align}
                
                \textbf{Solutions:}
                \begin{align}
                    A(t) &= A_0 e^{-k_1 t} \\
                    B(t) &= \frac{k_1 A_0}{k_2-k_1}(e^{-k_1 t} - e^{-k_2 t})
                \end{align}
            \end{block}
        \end{column}
        
        \begin{column}{0.5\textwidth}
            \begin{block}{Fractional Model (Our Approach)}
                \begin{align}
                    {}^C D_0^\alpha A(t) &= -k_1 A(t) \\
                    {}^C D_0^\beta B(t) &= k_1 A(t) - k_2 B(t)
                \end{align}
                
                \textbf{Solutions:}
                \begin{align}
                    A(t) &= A_0 E_\alpha(-k_1 t^\alpha) \\
                    B(t) &= \frac{A_0 k_1}{k_2-k_1}[E_\alpha(-k_1 t^\alpha) - E_\beta(-k_2 t^\beta)]
                \end{align}
                
                Where $E_\alpha(z)$ is the \highlight{Mittag-Leffler function}
            \end{block}
        \end{column}
    \end{columns}
    
    \vspace{0.3cm}
    \begin{center}
        \highlight{Key Parameters:} $\alpha = 0.8$ (absorption), $\beta = 0.9$ (elimination), $k_1 = 0.8$, $k_2 = 1.0$
    \end{center}
\end{frame}

\begin{frame}{Mittag-Leffler Function Implementation}
    \begin{block}{Mathematical Definition}
        The Mittag-Leffler function $E_\alpha(z)$ is defined as:
        \[
        E_\alpha(z) = \sum_{n=0}^{\infty} \frac{z^n}{\Gamma(\alpha n + 1)}
        \]
    \end{block}
    
    \begin{columns}
        \begin{column}{0.6\textwidth}
            \begin{block}{Numerical Implementation Challenges}
                \begin{itemize}
                    \item Series convergence for large negative arguments
                    \item Numerical stability with gamma function
                    \item Memory effects through fractional derivatives
                \end{itemize}
            \end{block}
            
            \begin{block}{Our Stable Implementation}
                \begin{itemize}
                    \item Asymptotic behavior for $z < -50$
                    \item Tolerance-based convergence ($10^{-15}$)
                    \item Overflow prevention mechanisms
                \end{itemize}
            \end{block}
        \end{column}
        
        \begin{column}{0.4\textwidth}
            \begin{center}
                \textcolor{kentech_blue}{\textbf{Mittag-Leffler}}\\
                \textcolor{kentech_blue}{\textbf{vs}}\\
                \textcolor{kentech_blue}{\textbf{Exponential}}\\
                \vspace{0.5cm}
                $E_\alpha(z)$ provides\\
                \highlight{memory effects}\\
                vs classical $e^z$
            \end{center}
        \end{column}
    \end{columns}
\end{frame}

\section{Implementation \& Applications}

\begin{frame}{Multi-Platform Application Suite}
    \begin{block}{Developed Applications}
        \begin{multicols}{2}
            \begin{enumerate}
                \item \highlight{Enhanced GUI Calculator}
                    \begin{itemize}
                        \item Modern tabbed interface
                        \item Real-time graph previews
                        \item Model comparison features
                    \end{itemize}
                
                \item \highlight{Web Application}
                    \begin{itemize}
                        \item Flask-based responsive design
                        \item Mobile-friendly interface
                        \item Interactive visualizations
                    \end{itemize}
                
                \item \highlight{Command Line Interface}
                    \begin{itemize}
                        \item Step-by-step input guidance
                        \item Batch processing capability
                        \item Quick calculations
                    \end{itemize}
                
                \item \highlight{Integrated Launcher}
                    \begin{itemize}
                        \item Dependency management
                        \item Application selection menu
                        \item System compatibility checks
                    \end{itemize}
            \end{enumerate}
        \end{multicols}
    \end{block}
    
    \begin{center}
        \textbf{Technology Stack:} Python, NumPy, SciPy, Matplotlib, Tkinter, Flask
    \end{center}
\end{frame}

\begin{frame}{User Input Parameters \& Calculations}
    \begin{columns}
        \begin{column}{0.5\textwidth}
            \begin{block}{Input Parameters}
                \textbf{Personal Information:}
                \begin{itemize}
                    \item Gender (Male/Female)
                    \item Age (19-100 years)
                    \item Weight (30-200 kg)
                    \item Height (optional)
                \end{itemize}
                
                \textbf{Drinking Information:}
                \begin{itemize}
                    \item Beverage type (Beer, Soju, Wine, etc.)
                    \item Volume consumed (mL)
                    \item Alcohol content (ABV \%)
                    \item Drinking start time
                \end{itemize}
            \end{block}
        \end{column}
        
        \begin{column}{0.5\textwidth}
            \begin{block}{Initial Concentration Calculation}
                \[
                A_0 = \frac{V \times \text{ABV} \times \rho_{\text{EtOH}}}{r \times m}
                \]
                Where:
                \begin{itemize}
                    \item $V$: Volume consumed (mL)
                    \item $\rho_{\text{EtOH}} = 0.789$ g/mL
                    \item $r$: TBW ratio (Male: 0.68, Female: 0.55)
                    \item $m$: Body weight (kg)
                \end{itemize}
            \end{block}
            
            \begin{block}{Safety Thresholds}
                \begin{itemize}
                    \item \warning{50 mg/100mL}: Driving limit
                    \item \safety{30 mg/100mL}: Safe driving
                    \item \safety{10 mg/100mL}: Complete recovery
                \end{itemize}
            \end{block}
        \end{column}
    \end{columns}
\end{frame}

\section{Results \& Analysis}

\begin{frame}{Model Comparison Results}
    \begin{center}
        \textcolor{kentech_blue}{\Large \textbf{BAC Model Comparison Results}}
    \end{center}
    
    \begin{block}{Key Findings}
        \begin{itemize}
            \item \highlight{Fractional model} shows more realistic memory effects
            \item \highlight{Slower initial rise} and more gradual decline in BAC
            \item \highlight{Better representation} of physiological alcohol metabolism
            \item \highlight{Improved accuracy} for recovery time predictions
        \end{itemize}
    \end{block}
    
    \begin{center}
        \textbf{Fractional model provides more conservative and safer predictions}
    \end{center}
\end{frame}

\begin{frame}{Recovery Time Analysis}
    \begin{columns}
        \begin{column}{0.6\textwidth}
            \begin{block}{Recovery Time Predictions}
                \textbf{Example: 70kg Male, 500mL Beer (5\% ABV)}
                
                \begin{tabular}{lcc}
                    \toprule
                    Threshold & Classical & Fractional \\
                    \midrule
                    Peak BAC & 0.064\% & 0.061\% \\
                    Peak Time & 1.2 hrs & 1.4 hrs \\
                    To 50mg/100mL & 3.8 hrs & 4.2 hrs \\
                    To 30mg/100mL & 4.5 hrs & 5.1 hrs \\
                    To 10mg/100mL & 6.2 hrs & 7.3 hrs \\
                    \bottomrule
                \end{tabular}
            \end{block}
            
            \begin{block}{Model Validation}
                \begin{itemize}
                    \item Fixed recovery time calculation bug
                    \item Improved Korean font rendering
                    \item Enhanced numerical stability
                \end{itemize}
            \end{block}
        \end{column}
        
        \begin{column}{0.4\textwidth}
            \begin{center}
                \textcolor{kentech_blue}{\textbf{Recovery Time}}\\
                \textcolor{kentech_blue}{\textbf{vs Weight}}\\
                \vspace{1cm}
                \highlight{Fractional model}\\
                predicts longer\\
                recovery times\\
                \vspace{0.5cm}
                \safety{Safer predictions}\\
                for all weight ranges
            \end{center>
        \end{column}
    \end{columns}
\end{frame}

\begin{frame}{Comprehensive Testing \& Validation}
    \begin{block}{Testing Scenarios}
        \begin{multicols}{2}
            \textbf{Beverage Types Tested:}
            \begin{itemize}
                \item Beer (4-6\% ABV)
                \item Soju (16-20\% ABV)
                \item Wine (12-15\% ABV)
                \item Whiskey (40\% ABV)
                \item Makgeolli (6-8\% ABV)
            \end{itemize}
            
            \textbf{Individual Variations:}
            \begin{itemize}
                \item Weight: 50-100kg
                \item Gender differences
                \item Age effects (19-65 years)
                \item Multiple drinking sessions
            \end{itemize}
        \end{multicols}
    \end{block}
    
    \begin{block}{Performance Metrics}
        \begin{itemize}
            \item \highlight{100\% success rate} in recovery time calculation
            \item \highlight{Stable numerical performance} across all test cases
            \item \highlight{Consistent results} between different applications
            \item \highlight{Physically realistic} BAC curves for all scenarios
        \end{itemize}
    \end{block}
\end{frame}

\section{Technical Achievements}

\begin{frame}{Key Technical Innovations}
    \begin{block}{Mathematical Contributions}
        \begin{enumerate}
            \item \highlight{Numerically stable Mittag-Leffler implementation}
                \begin{itemize}
                    \item Custom series computation with convergence control
                    \item Asymptotic behavior handling for extreme values
                    \item Memory-efficient calculation algorithms
                \end{itemize}
            
            \item \highlight{Fractional differential equation solver}
                \begin{itemize}
                    \item Caputo derivative implementation
                    \item Two-parameter Mittag-Leffler functions
                    \item Theoretical foundation validation
                \end{itemize}
            
            \item \highlight{Recovery time prediction algorithm}
                \begin{itemize}
                    \item Fixed peak detection logic
                    \item Accurate threshold crossing calculation
                    \item Realistic physiological constraints
                \end{itemize}
        \end{enumerate}
    \end{block}
\end{frame>

\begin{frame}{Software Engineering Excellence}
    \begin{columns}
        \begin{column}{0.5\textwidth}
            \begin{block}{Code Quality Features}
                \begin{itemize}
                    \item \highlight{Modular architecture} with reusable components
                    \item \highlight{Comprehensive error handling} and validation
                    \item \highlight{Extensive documentation} and user manuals
                    \item \highlight{Cross-platform compatibility} (Windows, macOS, Linux)
                \end{itemize}
            \end{block}
            
            \begin{block}{User Experience}
                \begin{itemize}
                    \item \highlight{Intuitive interfaces} for all skill levels
                    \item \highlight{Real-time visualization} of BAC curves
                    \item \highlight{Multiple export formats} (PNG, CSV, TXT)
                    \item \highlight{Multilingual support} (Korean/English)
                \end{itemize}
            \end{block}
        \end{column}
        
        \begin{column}{0.5\textwidth}
            \begin{block}{Testing \& Validation}
                \begin{itemize}
                    \item \highlight{Unit tests} for all mathematical functions
                    \item \highlight{Integration tests} for complete workflows
                    \item \highlight{Performance benchmarking} across platforms
                    \item \highlight{User acceptance testing} with multiple scenarios
                \end{itemize}
            \end{block}
            
            \begin{block}{Project Management}
                \begin{itemize}
                    \item \highlight{Version control} with comprehensive Git history
                    \item \highlight{Issue tracking} and resolution documentation
                    \item \highlight{Continuous improvement} through iterative development
                    \item \highlight{Complete project documentation}
                \end{itemize}
            \end{block}
        \end{column}
    \end{columns}
\end{frame}

\section{Conclusions \& Future Work}

\begin{frame}{Project Achievements}
    \begin{block}{Successfully Delivered}
        \begin{enumerate}
            \item \highlight{Complete mathematical framework} using fractional calculus
            \item \highlight{Four functional applications} with different interfaces
            \item \highlight{Accurate BAC prediction} with memory effect modeling
            \item \highlight{Practical recovery time estimation} for safety thresholds
            \item \highlight{Comprehensive testing suite} with validation results
        \end{enumerate}
    \end{block}
    
    \begin{block}{Impact \& Significance}
        \begin{itemize}
            \item \highlight{Advanced mathematical modeling} beyond traditional approaches
            \item \highlight{Practical safety applications} for alcohol consumption
            \item \highlight{Educational tool} for understanding fractional calculus
            \item \highlight{Foundation for further research} in physiological modeling
        \end{itemize}
    \end{block}
    
    \begin{block}{Academic Excellence}
        \begin{itemize}
            \item Demonstrates \highlight{mastery of engineering mathematics concepts}
            \item Shows \highlight{ability to solve real-world problems}
            \item Exhibits \highlight{software development and testing skills}
            \item Provides \highlight{clear documentation and presentation}
        \end{itemize}
    \end{block}
\end{frame}

\begin{frame}{Future Research Directions}
    \begin{block}{Model Enhancements}
        \begin{itemize}
            \item \highlight{Machine learning integration} for personalized parameters
            \item \highlight{Food effect modeling} with additional compartments
            \item \highlight{Time-varying coefficients} for dynamic rate adaptation
            \item \highlight{Multi-drug interaction} modeling capabilities
        \end{itemize}
    \end{block}
    
    \begin{block}{Application Extensions}
        \begin{itemize}
            \item \highlight{Mobile app development} for real-time monitoring
            \item \highlight{IoT integration} with breathalyzer devices
            \item \highlight{Social responsibility features} for group settings
            \item \highlight{Medical research applications} for clinical studies
        \end{itemize}
    \end{block>
    
    \begin{block}{Broader Impact}
        \begin{itemize}
            \item Potential for \highlight{public health policy} applications
            \item Integration into \highlight{driver education programs}
            \item Foundation for \highlight{smart city safety systems}
            \item Contribution to \highlight{alcohol harm reduction} strategies
        \end{itemize}
    \end{block}
\end{frame}

\begin{frame}{Thank You!}
    \begin{center}
        {\Large \textbf{Questions \& Discussion}}
        
        \vspace{1cm}
        
        \begin{block}{Project Repository}
            \textbf{Complete source code, documentation, and results available in:}\\
            \texttt{d:\textbackslash Kentech\textbackslash 2-1\textbackslash Engineering Mathematics 1\textbackslash EM\_Project}
        \end{block}
        
        \vspace{0.5cm}
        
        \begin{block}{Key Deliverables}
            \begin{itemize}
                \item Mathematical theory and implementation
                \item Four working applications (GUI, Web, CLI, Launcher)
                \item Comprehensive testing and validation
                \item Complete documentation and user manuals
                \item Research paper and technical reports
            \end{itemize}
        \end{block}
        
        \vspace{0.5cm}
        
        {\large \textcolor{kentech_blue}{\textbf{Engineering Mathematics 1 - Group 2}}}\\
        {\small Hyunjun Jang, Minyeop Jin, Sangsu Lee, Seojin Choi}
    \end{center}
\end{frame>

\end{document}
